% \usepackage[top=2cm, bottom=2cm, left=3cm, right=3cm]{geometry}
% \usepackage[T1]{fontenc}
% \usepackage[utf8]{inputenc}
% \usepackage[francais]{babel}
% \usepackage[backend=bibtex]{biblatex} %(a lot)Better than bibtex
% \usepackage{lastpage}           % Obtenir le nombre de pages
% \usepackage{fancyhdr}           % en-têtes et pieds de pages
% \usepackage{tabularx}           % tableaux
% \usepackage{multirow}             % fusionner les lignes d'un tableau
% \usepackage{media9}             % animations flash
% \usepackage{wrapfig}            %Incrustation d'images dans le texte
% \usepackage{graphicx}           %Inclure des images
% \usepackage{listingsutf8}       %Mise en forme des codes source
% \usepackage{url}                %Rendre cliquable les urls
% \usepackage{color}              %Pour utiliser des couleurs
% \usepackage{datatool}           %Permet de gérer .csv (non testé)
% \usepackage{csvsimple}          %Mise en forme tableau des .csv
% \usepackage{eurosym}            %Symbole €uro
% \usepackage{tikz}               %Pour dessiner
% \usepackage{tikz-qtree}         %Arbres généalogiques, binaires, ....
% \usepackage{makeidx}            %Création d'index
% \usepackage{CJKutf8}            %Japonais avec l'utf8
% \usepackage[overlap, CJK]{ruby} %Furigana
% \usepackage{amsmath}            %Better math environments
% \usepackage{amsfonts}           %Math symbols
% % ---------------------------------------------------------------
% % Parametrage pour le japonais
% \renewcommand{\rubysep}{-0.2ex} %Espace entre les furigana et les kanji
% \renewcommand{\rubysize}{0.7}   %Taille des furigana
% % ---------------------------------------------------------------
% % En-têtes et pieds de page
% \pagestyle{fancy}
% \renewcommand{\headrulewidth}{0.4pt}
% \renewcommand{\footrulewidth}{0.4pt}
% % \lhead{\slshape \leftmark}
% % \rhead{\slshape \rightmark}
% \rfoot{Note de pied de page}
% % ---------------------------------------------------------------
% \graphicspath{ {./images/} }    %Path for images
% \lstset{inputpath=./codes/}     %Path for source codes
% % ---------------------------------------------------------------
% % Mise en forme du code
% \lstset{
% language=C,
% basicstyle=\footnotesize,
% numbers=left,
% numberstyle=\tiny,
% numbersep=7pt,
% breaklines=true,
% backgroundcolor=\color{white},
% commentstyle=\color{green},
% stringstyle=\color{magenta},
% keywordstyle=\color{blue}
% }
% % ----------------------------------------------------------------
% \title[]{}
% \author{Prenom \bsc{Nom}}
% \date{\today}
%%%%%%%%%%%%%%%%%%
% Beamer Settings%
%%%%%%%%%%%%%%%%%%
% \usetheme{Warsaw}
% \title[]{}
% \subtitle{}
% \author{}
% \institute{}
% \logo{\includegraphics[width=1cm]{logo}}
% \date{\today}
% 
% \AtBeginSection[]          %TOC at sections beginning
% {
%   \begin{frame}
%     \frametitle{Table of Contents}
%     \tableofcontents[currentsection]
%   \end{frame}
% }
% \renewcommand{\footnotesize}{\scriptsize} %Smaller footnotes

%%%%%%%%%%%
% Biblatex%
%%%%%%%%%%%
% \addbibresource{$HOME/Bibliographies/bibfile}